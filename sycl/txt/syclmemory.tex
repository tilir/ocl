\documentclass[a4paper,12pt,oneside]{article}

\usepackage[utf8]{inputenc}
\usepackage[T2A]{fontenc}
\usepackage[english]{babel}
\usepackage[usenames]{xcolor}
\usepackage{etoolbox}
\usepackage{listings}
\usepackage{graphicx}
\usepackage{cmap}
\usepackage{indentfirst}
\usepackage{makeidx}
\usepackage[unicode]{hyperref}

\hypersetup{
%bookmarks=true,            % show bookmarks bar?
%unicode=false,             % non-Latin characters in Acrobat’s bookmarks
pdfproducer={Producer},    % producer of the document
pdfkeywords={keywords},    % list of keywords
pdfnewwindow=true,         % links in new window
colorlinks=true,           % false: boxed links; true: colored links
linkcolor=black,           % color of internal links
citecolor=black,           % color of links to bibliography
    filecolor=black,           % color of file links
    urlcolor=black             % color of external links
}

\definecolor{olivegreen}{cmyk}{0.64,0,0.95,0.40}
\definecolor{mauve}{rgb}{0.58,0,0.82}

\lstset{
language=C++,                           % Code langugage
basicstyle=\ttfamily,                   % Code font, Examples: \footnotesize, \ttfamily
keywordstyle=\color{olivegreen},        % Keywords font ('*' = uppercase)
commentstyle=\color{gray},              % Comments font
stringstyle=\color{mauve},
numbers=left,                           % Line nums position
numberstyle=\tiny,
numbersep=10pt,
stepnumber=1,                           % Step between two line-numbers
frame=none,                             % A frame around the code
tabsize=2,                              % Default tab size
captionpos=b,                           % Caption-position = bottom
breaklines=true,                        % Automatic line breaking?
breakatwhitespace=false,                % Automatic breaks only at whitespace?
showspaces=false,                       % Dont make spaces visible
showstringspaces=false,
showtabs=false,                         % Dont make tabls visible
columns=flexible,                       % Column format
title=\lstname,
caption={},
extendedchars=\true,
inputencoding=utf8,
}

\begin{document}
\section{SYCL memory basics}\label{sec:Basics}

% 1. Some common words about SYCL as generic system
% 2. We will dive deeper but first we need to understand SYCL model

\subsection{Host, device and shared memory}\label{subsec:HostDevice}

% 1. Explicit and implicit memory send back and forth
% 2. Shared memory may be really efficient for integrated GPUs

\subsection{Private, local and global memory}\label{subsec:PrivLocal}

% 1. SYCL accessors: global and local
% 2. Glorious GEMM example 
% 3. Private memory for explicit workgroup

\section{GPU hardware and how to think about memory}\label{sec:Basics}

% 1. Programming for GPU means (at least some) knowledge of architecture
% 2. Bad surprises may happen if you don't (example with too much private memory)

\subsection{Stateless, stateful and local memory}\label{subsec:StatelessFull}

% 1. Idea of binding tables and stateful buffers
% 2. Stateless memory is normal memory
% 3. Local memory is a chunk of explicit cache

\subsection{Scratch, TPM and registers}\label{subsec:ScratchTPM}

% 1. Register file and GRF size estimation
% 2. Per-thread memory with hardware support (scratch)
% 3. Thread priovate memory may be no better then global buffer

\section{Full stack and how to understand your compiler}\label{sec:Compiler}

% 1. Compilers play really huge role in modern compute world
% 2. Scheme of online and offline compilation

\subsection{Stateful to stateless transformation}\label{subsec:ToStateless}

% 1. When this transformation is possible and useful
% 2. How it happens and how to block it

\subsection{Generic address space resolution}\label{subsec:GenAddr}

% 1. Motivating synthetic case
% 2. Why generic addr spaces are disaster

\subsection{Workgroups are useful}\label{subsec:Wgroups}

% 1. Implicit vectorization between threads
% 2. Stunning example in bitonic sort

\subsection{Spill means no performance}\label{subsec:Spills}

% 1. Register allocation and spills
% 2. Measures

\subsection{Memory bank conflicts}\label{subsec:Banks}

% 1. Mental model of memory bank
% 2. What is bank conflict, example

\section{Vectorization and more}\label{sec:Vectors}

% 1. Explicit SIMD and problem of inner loop vectorization
% 2. Working inside subgroup means using the whole GRF (in theory)

\subsection{Gathering accesses and Laplace equation}\label{subsec:Banks}

% 1. Idea of gathers and scatters in hardware
% 2. Example of Laplace equation and making things better

\subsection{Explicit stateful memory}\label{subsec:Banks}

% 1. How to force stateful memory from the host code
% 2. Why do you want it

\subsection{Samplers for compute}\label{subsec:Banks}

% 1. Samplers and images are guests from 3D world: basic SYCL support
% 2. Can we sample any stateful buffer?

\end{document}
